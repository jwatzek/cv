% -*- program: xelatex -*-
\documentclass[]{friggeri-cv}
\usepackage{enumitem}
\usepackage{latexsym}
\usepackage{fontawesome}
\usepackage{graphicx}
\usepackage{fancyhdr}
\usepackage{soul}
\usepackage{hyperref}
	\fancyhf{} % clear all header and footers
	\renewcommand{\headrulewidth}{0pt} % remove the header rule
	\rfoot{curriculum vitae | \thepage}

\setlength{\footskip}{1.2cm}
\setlist{after=\vspace{6pt}}

\begin{document}
\newfontfamily{\FA}{FontAwesome}
\newcommand\faSkype{{\FA\symbol{"F17E}}}
\newcommand\faTw{{\FA\symbol{"F099}}}
\newcommand\faCode{{\FA\symbol{"F121}}}
\newcommand\faGit{{\FA\symbol{"F09B}}}
\newcommand\fahome{{\FA\symbol{"F015}}}
% http://fortawesome.github.io/Font-Awesome/cheatsheet/

\header{julia}{watzek}
       {curriculum vitae}

\raggedright

% In the aside, each new line forces a line break
\begin{aside}
  \section{contact}
    Dept of Psychology
    Language Research Ctr
    Georgia St University
    ~
    \href{mailto:j.watzek@gmail.com}{j.watzek@gmail.com}
    +1 (404) 840-6838
    ~
    \fahome{} \href{http://www.juliawatzek.com}{juliawatzek.com}
    \faTw{} \href{http://www.twitter.com/watzoever}{@watzoever}
    % \faSkype{} \href{Skype:j.watzek?call}{j.watzek}
    \faGit{} \href{https://github.com/jwatzek}{jwatzek}
  \section{~}
  \section{languages}
    german \& english bilingual $\cdot$ french \& spanish notions
  \section{coding}
    % {\color{red} $\varheartsuit$}
    \faCode{} R $\cdot$ Python $\cdot$ SAS $\cdot$ Matlab $\cdot$ Netlogo $\cdot$ \LaTeX\ $\cdot$ Java $\cdot$ XML $\cdot$ HTML $\cdot$ CSS $\cdot$ JavaScript
 % \section{~}
 % \section{orcid}
   % \includegraphics[width=3cm]{orcid_watzek}
\end{aside}

% \section{interests}
% decision-making $\cdot$ ``irrational'' behavior $\cdot$ cooperation $\cdot$ comparative cognition \\[-.1cm]
% collective behavior \\[-.1cm]

\section{education}

\begin{entrylist}
  \entry
    {2018--present}
    {Ph.\,D.~Candidate $\cdot$ Psychology\,/\,Cognitive Sciences}
    {Georgia State University}
    {\\[-.3cm]}
 \entry
   {2017}
   {M.\,A.~Psychology\,/\,Cognitive Sciences}
   {Georgia State University}
   {\\[-.3cm]}
  \entry
    {2012}
    {B.\,S.~Psychology}
    {Martin Luther University $\cdot$ Germany}
    {\\[-.3cm]}
\end{entrylist}

% =================================================
% GRANTS AND AWARDS
% =================================================

\section{grants \& awards}

\begin{entrylist}
  \entry
    {2016--present}
    {Duane M. Rumbaugh Fellowship Grant}
    {Georgia State University}
    {\\[-.3cm]}
  \entry
    {2019}
    {Outstanding Graduate Student Research Award}
    {Georgia State University}
    {\\[-.3cm]}
  \entry
    {2018}
    {Diversity Travel Award}
    {Animal Behavior Society}
    {\\[-.3cm]}
  \entry
    {2016}
    {Birnbaum Scholarship}
    {Society for Computers in Psychology}
    {\\[-.3cm]}
\end{entrylist}


% =================================================
% PUBLICATIONS
% =================================================

\section{publications}

% PUBLICATIONS: JOURNAL ARTICLES

\subsection{empirical articles}
\vspace{-.1cm}\hspace{.7cm}
{\small\addfontfeature{Color=lightgray} $\color{lightgray} {}^\bullet$ \emph{\color{lightgray} \ul{group authorship}} $\quad$ %\,/\,
{\normalsize $\color{lightgray} {}^\star$} \emph{\color{lightgray} equal contribution}}\\[.5cm]

\begin{enumerate}[label={[\,\arabic*\,]}]
  \item {Campbell MW\,{\Large${}^\star$}, \textbf{Watzek J\,{\Large${}^\star$}}, Suchak M, Berman SM, de Waal FBM (2020) Chimpanzees (\emph{Pan troglodytes}) tolerate some degree of inequity while cooperating but refuse to donate effort for nothing. \\\emph{Am J Primatol} 82:e23084. \href{https://doi.org/10.1002/ajp.23084}{\small $\nearrow$} }
  \item {Pope SM, Fagot J, Meguerditchian A, \textbf{Watzek J}, Lew-Levy S, Autrey MM, Hopkins WD (2020) Optional-switch cognitive flexibility in primates: Chimpanzees' (\emph{Pan troglodytes}) intermediate susceptibility to cognitive set. \\\emph{J Comp Psychol} 134:98–109. \href{https://doi.org/10.1037/com0000194}{\small $\nearrow$} }
  \item {\ul{Many Primates}\,${}^\bullet$ [31 authors] (2019) Establishing an infrastructure for collaboration in primate cognition research. \\\emph{PLOS ONE} 14:e0223675. \href{https://doi.org/10.1371/journal.pone.0223675}{\small $\nearrow$}}
  \item {\textbf{Watzek J}, Pope SM, Brosnan SF (2019) Capuchin and rhesus monkeys but not humans show cognitive flexibility in an optional-switch task. \\\emph{Sci Rep} 9:13195. \href{https://doi.org/10.1038/s41598-019-49658-0}{\small $\nearrow$}}
  \item {Aharoni E, Kleider-Offutt HM, Brosnan SF, \textbf{Watzek J} (2019) Justice at any cost? The impact of cost-benefit salience on criminal punishment judgments. \\\emph{Behav Sci Law} 37:38–60. \href{https://doi.org/10.1002/bsl.2388}{{\small $\nearrow$}}}
  \item {\textbf{Watzek J}, Whitham W, Washburn DA, Brosnan SF (2018) Responses to modified Monty Hall Dilemmas in capuchin monkeys, rhesus macaques, and humans. \\\emph{Int J Comp Psychol} 31. \href{https://escholarship.org/uc/item/1jn0t21r}{\small $\nearrow$}}
  \item {\textbf{Watzek J}, Brosnan SF (2018) (Ir)rational choices of humans, rhesus macaques, and capuchin monkeys in dynamic stochastic environments. \\\emph{Cognition} 178:109-117. \href{https://doi.org/10.1016/j.cognition.2018.05.019}{\small $\nearrow$}} \enlargethispage{1.75\baselineskip}
  \item {Amici F, Call J, \textbf{Watzek J}, Brosnan SF, Aureli F (2018) Social inhibition and behavioural flexibility when the context changes: a comparison across six primate species. \\\emph{Sci Rep} 8:3067. \href{https://doi.org/10.1038/s41598-018-21496-6}{\small $\nearrow$}}
  \item {Talbot CF, Parrish A, \textbf{Watzek J}, Essler JL, Leverett KL, Paukner A, Brosnan SF (2018) The influence of reward quality and quantity and spatial proximity on the responses to inequity and contrast in capuchin monkeys (\emph{Cebus [Sapajus] apella}). \\\emph{J Comp Psychol} 132:75-87. \href{https://doi.org/10.1037/com0000088}{\small $\nearrow$}}
  \item {Suchak M, \textbf{Watzek J}, Quarles LF, de Waal FBM (2018) Novice chimpanzees cooperate successfully in the presence of experts, but may have limited understanding of the task. \\\emph{Anim Cogn} 21:87–98. \href{https://doi.org/10.1007/s10071-017-1142-2}{\small $\nearrow$}}
  \item {Eppley TM, \textbf{Watzek J}, Hall K, Donati G (2017) Climatic, social and reproductive influences on behavioural thermoregulation in a female-dominated lemur. \\\emph{Anim Behav} 134:25-34. \href{https://doi.org/10.1016/j.anbehav.2017.10.003}{\small $\nearrow$}}
  \item {Eppley TM, \textbf{Watzek J}, Dausmann KH, Ganzhorn JU, Donati G (2017) Huddling is more important than rest site selection for thermoregulation in southern bamboo lemurs. \\\emph{Anim Behav} 127:153-161. \href{https://doi.org/10.1016/j.anbehav.2017.03.019}{\small $\nearrow$}}
  \item {Eppley TM, \textbf{Watzek J}, Ganzhorn JU, Donati G (2017) Predator avoidance and dietary fibre predict diurnality in the cathemeral folivore \emph{Hapalemur meridionalis}. \\\emph{Behav Ecol Sociobiol} 71:4. \href{https://doi.org/10.1007/s00265-016-2247-3}{\small $\nearrow$}} 
\end{enumerate}

\includegraphics[width=12.5cm]{author_contrib.png}\\[.7cm] %width=\linewidth
% \centerline{Author contributions}\\[.7cm]

% =====NEW=PAGE============================================

% \newpage
\pagestyle{fancy}

\renewenvironment{aside}{%
  \let\oldsection\section
  \renewcommand{\section}[1]{
    \par\vspace{\baselineskip}{\Large\headingfont\color{headercolor} ##1}
  }
  \begin{textblock}{3.6}(1.5, 1.5)
  \begin{flushright}
  \obeycr
}{%
  \restorecr
  \end{flushright}
  \end{textblock}
  \let\section\oldsection
}


\begin{aside}
  \section{{\normalfont julia}watzek}
    Dept of Psychology
    Language Research Ctr
    Georgia St University
    ~
    \href{mailto:j.watzek@gmail.com}{j.watzek@gmail.com}
    +1 (404) 840-6838
\end{aside}

% PUBLICATIONS: REVIEWS & COMMENTARY

\subsection{review articles}

\begin{enumerate}[resume, label={[\,\arabic*\,]}]
  \item \ul{Many Primates}\,${}^\bullet$ [18 authors] (\emph{in press}) Collaborative open science as a way to reproducibility and new insights in primate cognition research. \\\emph{Jpn Psychol Rev}. [Preprint] \href{https://doi.org/10.31234/osf.io/8w7zd}{\small $\nearrow$}
  \item {Smith MF, \textbf{Watzek J}, Brosnan SF (2018) The importance of a truly comparative methodology for comparative psychology. \\\emph{Int J Comp Psychol} 31. \href{https://escholarship.org/uc/item/6x91j98x}{\small $\nearrow$}}
\end{enumerate}

% PUBLICATIONS: BOOK CHAPTERS & ENCYCLOPEDIA ENTRIES

\subsection{book chapters}

\begin{enumerate}[resume, label={[\,\arabic*\,]}]
  \item Smith MF, \textbf{Watzek J}, Brosnan SF (\emph{in press}) Strategies used by non-human primates in dynamic games. In: Capra CM, Croson R, Rigdon MI, Rosenblat T (eds) \emph{Handbook of Experimental Game Theory}. Edward Elgar Publishing.
  \item \textbf{Watzek J}, Smith MF, Brosnan SF (2018) Comparative economics: Using experimental economic paradigms to understand primate social decision-making. In: di Paolo LD, di Vincenzo F, De Petrillo F (eds) \emph{Evolution of Primate Social Cognition}. Springer, p\,129-141. \href{https://doi.org/10.1007/978-3-319-93776-2_9}{\small $\nearrow$}
  \item \textbf{Watzek J}, Brosnan SF (2018) Role-reversal experiment. In: Vonk J, Shackelford T (eds) \emph{Encyclopedia of Animal Cognition and Behavior}. Springer. \href{https://doi.org/10.1007/978-3-319-47829-6_1497-1}{\small $\nearrow$}
\end{enumerate}

% =====NEW=PAGE============================================

\newpage

% PUBLICATIONS: SUBMITTED
% & WORKING PAPERS (commented out)

\subsection{submitted}

\SetLabelAlign{center}{\hss~~#1\hss}
% \renewcommand\labelitemi{$\Box$}
\begin{itemize}[align=center]
  \item \textbf{Watzek J}, Hauber ME, Jack KM, Murrell JR, Tecot SR, Brosnan SF (\emph{submitted}) The origins of collective decision-making: Insights into collective defense behavior from agent-based modeling.
  % \item \textbf{Watzek J}, Brosnan SF (\emph{in prep.}) Decision-making biases in animals: The fun is in the error bars.
  \item Aharoni E, Thurman J, Jahedi S, \textbf{Watzek J}, Fridlund AJ (\emph{submitted}) Thinking outside the (ballot) box: Effects of cost framing on simulated voting decisions.
  \item Cooper T\,{\Large${}^\star$}, Zabinski CL\,{\Large${}^\star$}, [\& 10 others] (\emph{submitted}) Long-term memory of a complex foraging task in squamate reptiles (Reptilia: Squamata: Varanidae).

  % \item \textbf{Watzek J}, Webster MF, Brosnan SF (\emph{in prep.}) Strategic decision-making and prediction of opponent behavior in monkeys and humans during a competitive two-player game.
  % \item \textbf{Watzek J}, Rossettie MS, Leverett K, Brosnan SF (\emph{in prep.}) Capuchin monkeys' (\emph{Cebus apella}) responses to inequity in a group setting.
\end{itemize}

% =================================================
% PRESENTATIONS
% =================================================

\section{presentations}

% PRESENTATIONS: TALKS

\subsection{talks}

\begin{entrylist2}
  \entrytwo
    {2019}{``Monkeys don't always follow the rules and neither should we: A comparative approach to studying cognitive flexibility.'' South Eastern Evolution and Human Behavior, Stone Mountain, GA.}
  \entrytwo
    {2019}{``Many Primates: A large-scale collaborative approach to studying primate cognition and behavior.'' American Society of Primatologists, Madison, WI.}
  \entrytwo
    {2018}{``Cognitive flexibility in humans and captive rhesus macaques (\emph{Macaca mulatta}) and capuchin monkeys (\emph{Cebus [Sapajus] apella}) in a simple problem-solving task.'' International Primatological Society, Nairobi, Kenya.}
  \entrytwo
    {2017}{``Strategic decision-making in capuchins (\emph{Cebus apella}), rhesus monkeys (\emph{Macaca mulatta}), and humans during a competitive two-player game.'' American Society of Primatologists, Washington, DC. [Published abstract] Am J Primatol 80(S1):39. \href{https://doi.org/10.1002/ajp.22942}{\small $\nearrow$}}
  \entrytwo
    {2016}{``Capuchin monkeys, but neither humans nor rhesus, are `irrational maximizers' in a stochastic environment.'' Comparative Cognition Society, Boston, MA.}
  \entrytwo
    {2015}{``Chimpanzees assess potential benefits before investing in a cooperative task.'' American Society of Primatologists, Bend, OR. [Published abstract] Am J Primatol 77(S1):90. \href{https://doi.org/10.1002/ajp.22494}{\small $\nearrow$}}
  \entrytwo
    {2015}{``More for me, less for you: Capuchin monkeys' (\emph{Cebus apella}) responses to inequity in a group setting.'' Animal Behavior Society, Anchorage, AK.}
  \entrytwo
    {2014}{``Contextual stability of behaviorally assessed personality traits in two groups of captive chimpanzees (\emph{Pan troglodytes}).'' American Society of Primatologists, Decatur, GA. [Published abstract] Am J Primatol 76(S1):79. \href{https://doi.org/10.1002/ajp.22382}{\small $\nearrow$}}
  \entrytwo
    {2014}{``Navigating pitfalls of researcher/caretaker collaboration.'' [Invited talk] Georgetown College Seminar: Cognition, Enrichment, and Collaboration, Zoo Atlanta, GA.}
\end{entrylist2}

\begin{aside}
  \section{{\normalfont julia}watzek}
    Dept of Psychology
    Language Research Ctr
    Georgia St University
    ~
    \href{mailto:j.watzek@gmail.com}{j.watzek@gmail.com}
    +1 (404) 840-6838
\end{aside}


% PRESENTATIONS: POSTERS

\subsection{posters}

\begin{entrylist2}
  \entrytwo
    {2019}{``Humans and captive capuchin monkeys (\emph{Cebus [Sapajus] apella}) and rhesus macaques (\emph{Macaca mulatta}) solve the Monty Hall Dilemma in a computerized task.'' American Society of Primatologists, Madison, WI.}
  \entrytwo
    {2018}{``WhatsOb: An Android app for note-taking during behavioral observations.'' International Primatological Society, Nairobi, Kenya.}
  \entrytwo
    {2018}{``Understanding collective defense behavior using agent-based modeling.'' Animal Behavior Society, Milwaukee, WI.}
  \entrytwo
    {2017}{``Justice at all costs? Transparency about the costs of incarceration decreases criminal sentencing recommendations by laypeople.'' Association for Psychological Science, Boston, MA.}
\entrytwo
    {2016}{``Rational fools: Capuchins but not rhesus monkeys violate transitivity to maximize their gains in stochastic environments.'' Psychonomic Society, Boston, MA. [Published abstract] Abstracts of the Psychonomic Society 21:172-173}
  \entrytwo
    {2016}{``WhatsOb: An Android app for note-taking during behavioral observations.'' Society for Computers in Psychology, Boston, MA.}
  \entrytwo
    {2016}{``Rational fools: Capuchin monkeys' (\emph{Cebus apella}) (ir)rational choices in stochastic environments.'' International Primatological Society and the American Society of Primatologists, Chicago, IL.}
  \entrytwo
    {2011}{``Less is more: Great apes' understanding of probabilities.'' Regional Conference of Developmental Psychologists, Halle (Saale), Germany.}
\end{entrylist2}

% =====NEW=PAGE============================================

% \newpage

% PRESENTATIONS: LOCAL

\subsection{departmental talks}

\begin{entrylist2}
  \entrytwo
    {2016}{``Rational fools: (Ir)rational choices of humans, rhesus macaques, and capuchin monkeys in dynamic stochastic environments.'' Cognitive Sciences Seminar, Georgia State University, Atlanta, GA.}
  \entrytwo
    {2015}{``More for me, less for you: Capuchin monkeys' (\emph{Cebus apella}) responses to inequity in a group setting.'' Cognitive Sciences Seminar, Georgia State University, Atlanta, GA.}
\end{entrylist2}

% PRESENTATIONS: LOCAL: POSTERS BY UNDERGRADS

\subsection{posters by undergraduate mentees}

\begin{entrylist2}
  \entrytwo
    {2018}{Gade PR. ``Responses to modified Monty Hall Dilemmas in capuchin monkeys, rhesus macaques, and humans.'' Psychology Undergraduate Research Conference, Georgia State University, Atlanta, GA.}
  \entrytwo
    {2017}{Le EH, Truscott CG. ``Chimpanzees' and capuchin monkeys' responses to inequity in a group setting.'' Georgia State Undergraduate Research Conference, Georgia State University, Atlanta, GA.\\
    {\quad\small\addfontfeature{Color=lightgray} $\color{lightgray} {}^\star$ \emph{\color{lightgray} Best Poster Award (Social and Behavioral Sciences)}}}
  \entrytwo
    {2016}{Fakhri A, Le EH. ``Capuchin monkeys respond to advantageous inequity.'' Psychology Undergraduate Research Conference, Georgia State University, Atlanta, GA.}
\end{entrylist2}

% =====NEW=PAGE============================================

% \newpage

\begin{aside}
  \section{{\normalfont julia}watzek}
    Dept of Psychology
    Language Research Ctr
    Georgia St University
    ~
    \href{mailto:j.watzek@gmail.com}{j.watzek@gmail.com}
    +1 (404) 840-6838
\end{aside}

% =================================================
% RESEARCH
% =================================================

\section{research}

\begin{entrylist}
  \entry
    {2018--present}
    {ManyPrimates}
    {\href{https://manyprimates.github.io}{manyprimates.github.io}}
    {Contributor $\cdot$ Project Coordinator (2019--present)}
  \entry
    {2018--present}
    {Herpetology Department}
    {Zoo Atlanta}
    {Graduate Student Researcher (PI: Joe Mendelson)}
  \entry
    {2014--present}
    {Language Research Center}
    {Georgia State University}
    {CEBUS Lab $\cdot$ Graduate Student Researcher (PI: Sarah Brosnan)}
  \entry
    {2018}
    {Evolution, Variation, and Ontogeny of Learning Lab}
    {University of Texas at Austin}
    {Visiting Research Scholar (PI: Cristine Legare) $\cdot$ Fieldwork Manipur, India}
  \entry
    {2016--2018}
    {Cooperation, Conflict, \& Cognition Lab}
    {Georgia State University}
    {Graduate Student Researcher (PI: Eyal Aharoni)}
  \entry
    {2012--2014}
    {Yerkes National Primate Research Center}
    {Emory University}
    {Living Links Center $\cdot$ Lab Manager $\cdot$ Chimpanzee Research Specialist}
  \entry
    {2011--2012}
    {Inkawu Vervet Project $\cdot$ South Africa}
    {U's of St Andrews, Neuch\^{a}tel, Z\"urich, \& Cape Town}
    {Field Research Assistant}
  \entry
    {2010--2011}
    {Wolfgang K\"{o}hler Primate Research Center}
    {Zoo Leipzig $\cdot$ Germany}
    {Research Assistant $\cdot$ Thesis Student}
  \entry
    {2010--2011}
    {Educational Psychology Lab}
    {Martin Luther University $\cdot$ Germany}
    {Lab Coordinator}
  \entry
    {2010}
    {Child \& Adolescent Psychiatry \& Psychotherapy}
    {St Elisabeth Hospital Halle $\cdot$ Germany}
    {Intern}
  \entry
    {2009--2010}
    {Psychology of Perception Lab}
    {Martin Luther University $\cdot$ Germany}
    {Research Assistant}
  \entry
    {2009--2010}
    {Social Psychology Lab}
    {Martin Luther University $\cdot$ Germany}
    {Research Assistant}
\end{entrylist}


% =====NEW=PAGE============================================

% \newpage

\begin{aside}
  \section{{\normalfont julia}watzek}
    Dept of Psychology
    Language Research Ctr
    Georgia St University
    ~
    \href{mailto:j.watzek@gmail.com}{j.watzek@gmail.com}
    +1 (404) 840-6838
\end{aside}


% =================================================
% TEACHING
% =================================================

\section{teaching}

\subsection{instructor of record}

\begin{entrylist}
  \entry
    {Spr 2020}
    {Advanced Research Design \& Analysis (Lab)}
    {Georgia State University}
    {Advanced Undergraduate}
  \entry
    {Spr 2019}
    {Introduction to Research Design \& Analysis}
    {Georgia State University}
    {Advanced Undergraduate}
  \entry
    {Fall 2017}
    {Introduction to General Psychology}
    {Georgia State University}
    {Undergraduate}
  \entry
    {Win 2009}
    {Introduction to R for Statistics}
    {Martin Luther University $\cdot$ Germany}
    {Undergraduate}
\end{entrylist}

\subsection{teaching assistant}

\begin{entrylist}
  \entry
    {Sum 2010/12}
    {Quantitative Methods II}
    {Martin Luther University $\cdot$ Germany}
    {Undergraduate}
  \entry
    {Win 2010}
    {Quantitative Methods I}
    {Martin Luther University $\cdot$ Germany}
    {Undergraduate}
\end{entrylist}

\subsection{guest lecturer}

\begin{entrylist}
  \entry
    {Spr 2018}
    {Primate Models of Human Behavior}
    {Georgia State University}
    {Advanced Undergraduate: Cooperation}
  \entry
    {Spr 2018}
    {Introduction to Research Design \& Analysis}
    {Georgia State University}
    {Advanced Undergraduate: z-Scores}
  \entry
    {Fall 2015/16,\\Spr 2018\\[-.85cm]}
    {Psychology of Animal Behavior}
    {Georgia State University}
    {Advanced Undergraduate: Aggression $\cdot$ Cooperation $\cdot$ Comparative \\
    Economics $\cdot$ Monkey Business: Comparative Economics and \\
    Decision-Making}
  \entry
    {Spr 2014}
    {Primate Social Psychology}
    {Emory University}
    {Advanced Undergraduate: Social Complexity of Non-Primates}
\end{entrylist}


% =================================================
% AD HOC REVIEWER
% =================================================

\section{ad~hoc reviewer}
    Am J Primatol $\cdot$ Biol Lett $\cdot$ J Comp Psychol $\cdot$ PLOS ONE \\[.5cm]


% =================================================
% PROFESSIONAL SOCIETIES
% =================================================

\section{professional societies}
    American Society of Primatologists $\cdot$ %\\
    Animal Behavior Society $\cdot$ %\\
    APA Division 3: Society for Experimental Psychology and Cognitive Science $\cdot$ %\\
    APA Division 6: Society for Behavioral Neuroscience and Comparative Psychology $\cdot$ %\\
    % Association for Psychological Science $\cdot$ %\\
    Comparative Cognition Society $\cdot$ %\\
    International Primatological Society $\cdot$ %\\
    % Psi Chi $\cdot$ %\\
    Psychonomic Society \\[.5cm]%$\cdot$ %\\
    % Society for Computers in Psychology \\[.5cm]


% =====NEW=PAGE============================================

\newpage

\begin{aside}
  \section{{\normalfont julia}watzek}
    Dept of Psychology
    Language Research Ctr
    Georgia St University
    ~
    \href{mailto:j.watzek@gmail.com}{j.watzek@gmail.com}
    +1 (404) 840-6838
\end{aside}


% =================================================
% OTHER ACTIVITIES
% =================================================
\section{other activities}

\begin{entrylist}
  \entry
    {2018--present}
    {Hard Data Café}
    {Georgia State University}
    {Cognitive Sciences Seminar Steering Committee}
  \entry
    {2014--2015}
    {Cebus Lab Website}
    {Georgia State University}
    {Webmaster}
  \entry
    {2012--2014}
    {Living Links Website}
    {Emory University}
    {Webmaster}
  \entry
    {2008--2012}
    {Martin Luther University}
    {}
    {Student Representative}
 \entry
   {2011}
   {Conference of Experimental Psychologists}
   {Germany}
   {Conference Assistant}
 \entry
   {2011}
   {Regional Conference of Developmental Psychologists}
   {Central Germany}
   {Conference Assistant}
 \entry
   {2010}
   {Martin Luther University}
   {}
   {Electoral Assistant}
\end{entrylist}


% =================================================
% OUTREACH
% =================================================

\section{outreach}

\begin{entrylist}
  \entry
    {2016, 2017}
    {Adventure in Science Day}
    {Fernbank Museum of Natural History}
    {Science outreach for children $\cdot$ Atlanta, GA}
  \entry
    {2015, 2017}
    {Brain Awareness Campaign}
    {Society for Neuroscience}
    {K-12 classroom visits $\cdot$ Atlanta, GA}
  \entry
    {2013}
    {Nerd Nite}
    {}
    {Presentation $\cdot$ ``We are 98\% chimp and 50\% bananas: What apes can teach us about our own shortcomings'' $\cdot$ Atlanta, GA}
  \entry
    {2012--2014}
    {Tour Guide}
    {Emory University}
    {Yerkes National Primate Research Center}
  \entry
    {2010--2011}
    {Tour Guide}
    {Zoo Leipzig $\cdot$ Germany}
    {Wolfgang K\"{o}hler Primate Research Center}
\end{entrylist}


% =================================================
% SELECTED MEDIA COVERAGE
% =================================================

\section{selected media coverage}
\vspace{-.1cm}
{\small\addfontfeature{Color=lightgray} My research has also been covered in other languages, including Chinese, Dutch,  French, German, Italian, Norwegian, Portuguese, Russian, and Spanish.}\\[.7cm]

\begin{entrylist}
  \entry
    {Dec 2019}
    {Psychology Today \href{https://www.psychologytoday.com/us/blog/wild-connections/201912/3-ways-improve-your-cognitive-flexibility}{\small $\nearrow$}}
    {}
    {``3 ways to improve your cognitive flexibility''}
  \entry
    {Oct 2019}
    {idw Informationsdienst Wissenschaft (scientific information service) \href{https://idw-online.de/de/news726094}{\small $\nearrow$}}
    {}
    {``Leipzig primate researchers initiate global collaboration''}
  \entry
    {Oct 2019}
    {Global News Radio}
    {}
    {{[}Radio{]} ON Point with Alex Pierson}
  \entry
    {Oct 2019}
    {IFLScience \href{https://www.iflscience.com/brain/brainy-monkeys-outsmart-humans-in-cognitive-flexibility-experiment/}{\small $\nearrow$}}
    {}
    {``Brainy monkeys outsmart humans in cognitive flexibility experiment''}
  \entry
    {Oct 2019}
    {ScienceDaily \href{https://www.sciencedaily.com/releases/2019/10/191015115356.htm}{\small $\nearrow$}}
    {}
    {``Monkeys outperform humans when it comes to cognitive flexibility''}
  \entry
    {Oct 2019}
    {Live Science \href{https://www.livescience.com/monkeys-outsmart-humans.html}{\small $\nearrow$}}
    {}
    {``Game over: These monkeys just crushed humans on a computer game''}
  \entry
    {Oct 2019}
    {Daily Mail \href{https://www.dailymail.co.uk/sciencetech/article-7585197/Monkeys-OUTSMART-humans-problem-solving-exercise-win-food-test-cognitive-flexibility.html}{\small $\nearrow$}}
    {}
    {``Monkeys OUTSMART humans in problem solving exercise to win food in test of cognitive flexibility''}
  \entry
    {Oct 2019}
    {Daily Star \href{https://www.dailystar.co.uk/news/weird-news/monkeys-beat-humans-computer-game-20642153}{\small $\nearrow$}}
    {}
    {``Monkeys beat humans at computer game in groundbreaking intelligence study''}
  \entry
    {Jun 2019}
    {Science Trends \href{https://sciencetrends.com/lemurs-groom-high-ranking-females-in-order-to-gain-thermoregulatory-benefits/}{\small $\nearrow$}}
    {}
    {``Lemurs groom high-ranking females in order to gain thermoregulatory benefits''}
  \entry
    {Jan 2019}
    {Georgia Public Broadcasting (GPB) Radio News \href{https://www.gpbnews.org/post/lock-em-any-cost-ga-study-suggests-most-wouldnt}{\small $\nearrow$}}
    {}
    {``Lock 'em up at any cost? Ga. study suggests most wouldn't''}
  \entry
    {Jan 2019}
    {Phys.org \href{https://phys.org/news/2019-01-people-temper-criminal-sentences-incarceration.html}{\small $\nearrow$}}
    {}
    {``People likely to temper criminal sentences when given information about the cost of incarceration''}
\end{entrylist}


% =====NEW=PAGE============================================

% \newpage

\begin{aside}
  \section{{\normalfont julia}watzek}
    Dept of Psychology
    Language Research Ctr
    Georgia St University
    ~
    \href{mailto:j.watzek@gmail.com}{j.watzek@gmail.com}
    +1 (404) 840-6838
\end{aside}





% =================================================
% REFERENCES
% =================================================

% ...

\end{document}
