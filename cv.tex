% -*- program: xelatex -*-
\documentclass[]{friggeri-cv}
\usepackage{fontawesome}
\usepackage{graphicx}
\usepackage{fancyhdr}
	\fancyhf{} % clear all header and footers
	\renewcommand{\headrulewidth}{0pt} % remove the header rule
	\rfoot{curriculum vitae | \thepage}

\begin{document}
\newfontfamily{\FA}{FontAwesome}
\newcommand\faSkype{{\FA\symbol{"F17E}}}
\newcommand\faTw{{\FA\symbol{"F099}}}
\newcommand\faCode{{\FA\symbol{"F121}}}
\newcommand\faGit{{\FA\symbol{"F09B}}}
\newcommand\fahome{{\FA\symbol{"F015}}}
% http://fortawesome.github.io/Font-Awesome/cheatsheet/

\header{julia}{watzek}
       {curriculum vitae}

\raggedright

% In the aside, each new line forces a line break
\begin{aside}
  \section{contact}
    Dept of Psychology
    Language Research Ctr
    Georgia St University
    ~
    \href{mailto:j.watzek@gmail.com}{j.watzek@gmail.com}
    +1 (404) 840-6838
    ~
    \fahome{} \href{http://www.juliawatzek.com}{juliawatzek.com}
    \faTw{} \href{http://www.twitter.com/juliawatzek}{@juliawatzek}
    % \faSkype{} \href{Skype:j.watzek?call}{j.watzek}
    \faGit{} \href{https://github.com/jwatzek}{jwatzek}
  \section{~}
  \section{languages}
    german \& english bilingual $\cdot$ french \& spanish notions
  \section{coding}
    % {\color{red} $\varheartsuit$}
    \faCode{} R $\cdot$ Python $\cdot$ SAS $\cdot$ Matlab $\cdot$ Netlogo $\cdot$ \LaTeX\ $\cdot$ Java $\cdot$ XML $\cdot$ HTML $\cdot$ CSS $\cdot$ JavaScript
%  \section{~}
%  \section{orcid}
%    \includegraphics[width=3cm]{orcid_watzek}
\end{aside}

% \section{interests}
% decision-making $\cdot$ ``irrational'' behavior $\cdot$ cooperation $\cdot$ comparative cognition \\[-.1cm]
% social networks $\cdot$ collaborative decision-making $\cdot$ conformity to misinformation $\cdot$ irrational choice $\cdot$ cooperation $\cdot$ inequity aversion \\[-.1cm]

\section{education}

\begin{entrylist}
  \entry
    {2014--present}
    {Ph.\,D.~Student $\cdot$ Psychology\,/\,Cognitive Sciences}
    {Georgia State University}
    {\\[-.7cm]}
 \entry
   {2017}
   {M.\,A.~Psychology\,/\,Cognitive Sciences}
   {Georgia State University}
   {\\[-.7cm]}
  \entry
    {2012}
    {B.\,S.~Psychology}
    {Martin Luther University $\cdot$ Germany}
    {}
\end{entrylist}\\[-.1cm]

% =================================================
% GRANTS AND AWARDS
% =================================================

\section{grants \& awards}

\begin{entrylist}
  \entry
    {2016--present}
    {Duane M. Rumbaugh Fellowship Grant}
    {Georgia State University}
    {\\[-.75cm]}
  \entry
    {2016}
    {Birnbaum Scholarship}
    {Society for Computers in Psychology}
    {}
\end{entrylist}\\[-.1cm]

% =================================================
% PUBLICATIONS
% =================================================

\section{publications}

% % PUBLICATIONS: JOURNAL ARTICLES

% {\subfont\large\addfontfeature{Color=lightgray} journal articles}

% \begin{entrylist2}
%   \entrytwo
%     {\emph{accepted}}
%     {Eppley TM, \textbf{Watzek J}, Hall K, Donati G (\emph{accepted with revisions}) Grooming `up' a female-dominated social hierarchy facilitates thermoregulation. Anim Behav.}
%   \entrytwo
%     {\emph{in press}}
%     {Talbot CF, Parrish A, \textbf{Watzek J}, Essler JL, Leverett KL, Paukner A, Brosnan SF (\emph{in press}) The influence of reward quality and quantity and spatial proximity on the responses to inequity and contrast in capuchin monkeys (\emph{Cebus [Sapajus] apella}). J~Comp Psychol.}
%   \entrytwo
%     {2017}
%     {Eppley TM, \textbf{Watzek J}, Dausmann KH, Ganzhorn JU, Donati G (2017) Huddling is more important than rest site selection for thermoregulation in southern bamboo lemurs. Anim Behav 127:153-161. doi:10.1016/j.anbehav.2017.03.019}
%   \entrytwo
%     {2017}
%     {Eppley TM, \textbf{Watzek J}, Ganzhorn JU, Donati G (2017) Predator avoidance and dietary fibre predict diurnality in the cathemeral folivore \emph{Hapalemur meridionalis}. Behav Ecol Sociobiol 71:4. doi:10.1007/s00265-016-2247-3\\}

%   % PUBLICATIONS: PEER-REVIEWED: SUBMITTED
%   \entrytwo
%     {}
%     {\rule{0.5\textwidth}{.4pt}\\[.2cm]}
%   \entrytwo
%     {\emph{submitted}}
%     {\textbf{Watzek J}, Brosnan SF (\emph{submitted}) (Ir)rational choices of humans, rhesus macaques, and capuchin monkeys in dynamic stochastic environments.}
%   \entrytwo
%     {\emph{submitted}}
%     {Amici F, Call J, \textbf{Watzek J}, Brosnan SF, Aureli F (\emph{submitted}) When the social context changes: Social inhibition and behavioural flexibility.}
%   \entrytwo
%     {\emph{submitted}}
%     {Suchak M, \textbf{Watzek J}, Quarles LF, de Waal FBM (\emph{submitted}) Novice chimpanzees cooperate successfully in the presence of experts, but with limited understanding and efficiency.}

%   % PUBLICATIONS: PEER-REVIEWED: WORKING PAPERS
%   % \entrytwo
%   %   {\emph{in prep.}}
%   %   {\textbf{Watzek J}, Smith MF, Brosnan SF (\emph{in prep.}) Strategic decision-making and prediction of opponent behavior in monkeys and humans during a competitive two-player game}
% \end{entrylist2}

% % PUBLICATIONS: BOOK CHAPTERS

% {\subfont\large\addfontfeature{Color=lightgray} book chapters}

% % PUBLICATIONS: CHAPTERS: SUBMITTED

% \begin{entrylist2}
%   \entrytwo
%     {\emph{submitted}}
%     {\textbf{Watzek J}, Smith MF, Brosnan SF (\emph{submitted}) Comparative economics: Using experimental economics paradigms to understand primate social decision-making. In: d'Almeida AF, di Paolo LD, di Vincenzo F (eds) Evolution of Primate Social Cognition. Springer.}
%   \entrytwo
%     {\emph{submitted}}
%     {Smith MF, \textbf{Watzek J}, Brosnan SF (\emph{submitted}) Strategies used by non-human primates in dynamic games. In: Croson R, Capra M, Rigdon MI, Rosenblat T (eds) Handbook of Experimental Game Theory. Edward Elgar Publishing.}
% \end{entrylist2}\\[-.5cm]


% PUBLICATIONS: JOURNAL ARTICLES

{\subfont\large\addfontfeature{Color=lightgray} journal articles}

\hangindent=.7cm Suchak M, \textbf{Watzek J}, Quarles LF, de Waal FBM (\emph{in press}) Novice chimpanzees cooperate successfully in the presence of experts, but may have limited understanding of the task. Anim Cogn.

\hangindent=.7cm Eppley TM, \textbf{Watzek J}, Hall K, Donati G (\emph{in press}) Climatic, social, and reproductive influences on behavioural thermoregulation in a female-dominated lemur. Anim Behav.

\hangindent=.7cm Talbot CF, Parrish A, \textbf{Watzek J}, Essler JL, Leverett KL, Paukner A, Brosnan SF (\emph{in press}) The influence of reward quality and quantity and spatial proximity on the responses to inequity and contrast in capuchin monkeys (\emph{Cebus [Sapajus] apella}). J~Comp Psychol.

\hangindent=.7cm Eppley TM, \textbf{Watzek J}, Dausmann KH, Ganzhorn JU, Donati G (2017) Huddling is more important than rest site selection for thermoregulation in southern bamboo lemurs. Anim Behav 127:153-161. doi:10.1016/j.anbehav.2017.03.019

\hangindent=.7cm Eppley TM, \textbf{Watzek J}, Ganzhorn JU, Donati G (2017) Predator avoidance and dietary fibre predict diurnality in the cathemeral folivore \emph{Hapalemur meridionalis}. Behav Ecol Sociobiol 71:4. doi:10.1007/s00265-016-2247-3 \\[.5cm]

% PUBLICATIONS: PEER-REVIEWED: SUBMITTED

\hspace{.35cm} {\addfontfeature{Color=gray} \underline{submitted}}

\hangindent=.7cm \textbf{Watzek J}, Brosnan SF (\emph{submitted}) (Ir)rational choices of humans, rhesus macaques, and capuchin monkeys in dynamic stochastic environments.

\hangindent=.7cm Amici F, Call J, \textbf{Watzek J}, Brosnan SF, Aureli F (\emph{submitted}) Making decisions when the social context changes: Social inhibition and behavioural flexibility.

\hangindent=.7cm Aharoni E, Jahedi S, Haskell A, \textbf{Watzek J}, Parker A, Fridlund A (\emph{submitted}) Thinking outside the (ballot) box: Effects of cost framing on simulated voting decisions. \\[.5cm]

% PUBLICATIONS: PEER-REVIEWED: WORKING PAPERS



% PUBLICATIONS: BOOK CHAPTERS

{\subfont\large\addfontfeature{Color=lightgray} book chapters}

% PUBLICATIONS: CHAPTERS: SUBMITTED

\hspace{.35cm} {\addfontfeature{Color=gray} \underline{submitted}}

\hangindent=.7cm \textbf{Watzek J}, Smith MF, Brosnan SF (\emph{submitted}) Comparative economics: Using experimental economics paradigms to understand primate social decision-making. In: d'Almeida AF, di Paolo LD, di Vincenzo F (eds) Evolution of Primate Social Cognition. Springer.

% =====NEW=PAGE============================================

\newpage
\thispagestyle{fancy}

\hangindent=.7cm Smith MF, \textbf{Watzek J}, Brosnan SF (\emph{submitted}) Strategies used by non-human primates in dynamic games. In: Croson R, Capra M, Rigdon MI, Rosenblat T (eds) Handbook of Experimental Game Theory. Edward Elgar Publishing.\\[.4cm]

% PUBLICATIONS: CHAPTERS: WORKING PAPER

% \hangindent=.7cm \textbf{Watzek J}, Brosnan SF (\emph{in prep.}) Role-reversal experiment. In: Vonk J, Shackelford TK (eds) Encyclopedia of Animal Cognition and Behavior. Springer.


\renewenvironment{aside}{%
  \let\oldsection\section
  \renewcommand{\section}[1]{
    \par\vspace{\baselineskip}{\Large\headingfont\color{headercolor} ##1}
  }
  \begin{textblock}{3.6}(1.5, 1.5)
  \begin{flushright}
  \obeycr
}{%
  \restorecr
  \end{flushright}
  \end{textblock}
  \let\section\oldsection
}


\begin{aside}
  \section{{\normalfont julia}watzek}
    Dept of Psychology
    Language Research Ctr
    Georgia St University
    ~
    \href{mailto:j.watzek@gmail.com}{j.watzek@gmail.com}
    +1 (404) 840-6838
\end{aside}


% =================================================
% PRESENTATIONS
% =================================================

\section{presentations}

% PRESENTATIONS: TALKS

{\subfont\large\addfontfeature{Color=lightgray} talks}

\begin{entrylist2}
  \entrytwo
    {2017}
    {``Strategic decision-making in capuchins (\emph{Cebus apella}), rhesus monkeys (\emph{Macaca mulatta}), and humans during a competitive two-player game.'' American Society of Primatologists, Washington, DC.}
  \entrytwo
    {2016}
    {``Capuchin monkeys, but neither humans nor rhesus, are `irrational maximizers' in a stochastic environment.'' Comparative Cognition Society, Boston, MA.}
  \entrytwo
    {2015}
    {``Chimpanzees assess potential benefits before investing in a cooperative task.'' American Society of Primatologists, Bend, OR. [Published abstract] Am J Primatol 77(S1):90. doi:10.1002/ajp.22494}
  \entrytwo
    {2015}
    {``More for me, less for you: Capuchin monkeys' (\emph{Cebus apella}) responses to inequity in a group setting.'' Animal Behavior Society, Anchorage, AK.}
  \entrytwo
    {2014}
    {``Contextual stability of behaviorally assessed personality traits in two groups of captive chimpanzees (\emph{Pan troglodytes}).'' American Society of Primatologists, Decatur, GA. [Published abstract] Am J Primatol 76(S1):79. doi:10.1002/ajp.22382}
  \entrytwo
    {2014}
    {``Navigating pitfalls of researcher/caretaker collaboration.'' [Invited talk] Georgetown College Seminar: Cognition, Enrichment, and Collaboration, Zoo Atlanta, GA.}
\end{entrylist2}

% PRESENTATIONS: POSTERS

{\subfont\large\addfontfeature{Color=lightgray} posters}

\begin{entrylist2}
  \entrytwo
    {2017}
    {``Justice at all costs? Transparency about the costs of incarceration decreases criminal sentencing recommendations by laypeople.'' Association for Psychological Science, Boston, MA.}
  \entrytwo
    {2016}
    {``Rational fools: Capuchins but not rhesus monkeys violate transitivity to maximize their gains in stochastic environments.'' Psychonomic Society, Boston, MA. [Published abstract] Abstracts of the Psychonomic Society 21:172-173}
  \entrytwo
    {2016}
    {``WhatsOb: An Android app for note-taking during behavioral observations.'' Society for Computers in Psychology, Boston, MA.}
  \entrytwo
    {2016}
    {``Rational fools: Capuchin monkeys' (\emph{Cebus apella}) (ir)rational choices in stochastic environments.'' International Primatological Society and the American Society of Primatologists, Chicago, IL.}
  \entrytwo
    {2011}
    {``Less is more: Great apes' understanding of probabilities.'' Regional Conference of Developmental Psychologists, Halle (Saale), Germany.}
\end{entrylist2}

% PRESENTATIONS: LOCAL

{\subfont\large\addfontfeature{Color=lightgray} departmental talks}

\begin{entrylist2}
  \entrytwo
    {2016}
    {``Rational fools: (Ir)rational choices of humans, rhesus macaques, and capuchin monkeys in dynamic stochastic environments.'' Cognitive Sciences Seminar, Georgia State University, Atlanta, GA.}
  \entrytwo
    {2015}
    {``More for me, less for you: Capuchin monkeys' (\emph{Cebus apella}) responses to inequity in a group setting.'' Cognitive Sciences Seminar, Georgia State University, Atlanta, GA.}
\end{entrylist2}

% =====NEW=PAGE============================================

\newpage
\thispagestyle{fancy}

\begin{aside}
  \section{{\normalfont julia}watzek}
    Dept of Psychology
    Language Research Ctr
    Georgia St University
    ~
    \href{mailto:j.watzek@gmail.com}{j.watzek@gmail.com}
    +1 (404) 840-6838
\end{aside}

% PRESENTATIONS: LOCAL: POSTERS BY UNDERGRADS

{\subfont\large\addfontfeature{Color=lightgray} posters by undergraduate mentees}

\begin{entrylist2}
  \entrytwo
    {2017}
    {Le EH, Truscott CG. ``Chimpanzees' and capuchin monkeys' responses to inequity in a group setting.'' Georgia State Undergraduate Research Conference, Georgia State University, Atlanta, GA.\\
    \emph{${}^\star$ Best Poster Award for the Social and Behavioral Sciences}}
  \entrytwo
    {2016}
    {Fakhri A, Le EH. ``Capuchin monkeys respond to advantageous inequity.'' Psychology Undergraduate Research Conference, Georgia State University, Atlanta, GA.}
\end{entrylist2}


% =================================================
% TEACHING
% =================================================

\section{teaching}

\begin{entrylist}
  \entry
    {Fall 2017}
    {Introduction to General Psychology}
    {Georgia State University}
    {Instructor of Record $\cdot$ Undergraduate}
  \entry
    {Fall 2015/16}
    {Psychology of Animal Behavior}
    {Georgia State University}
    {Guest Lecturer $\cdot$ Advanced Undergraduate:\\ 
    $\cdot$ Cooperation\\
    $\cdot$ Comparative Economics\\
    $\cdot$ Monkey Business: Comparative Economics and Decision-Making}
  \entry
    {Spr 2014}
    {Primate Social Psychology}
    {Emory University}
    {Guest Lecturer $\cdot$ Advanced Undergraduate: Social Complexity of\\ Non-Primates}
  \entry
    {Sum 2010/12,\\Win 2010\\[-.85cm]}
    {Quantitative Methods in Psychology I \& II}
    {Martin Luther University $\cdot$ Germany}
    {Teaching Assistant $\cdot$ Undergraduate}
  \entry
    {Win 2009}
    {Introduction to R for Statistics}
    {Martin Luther University $\cdot$ Germany}
    {Instructor of Record $\cdot$ Undergraduate}
\end{entrylist}

% =================================================
% RESEARCH EXPERIENCE
% =================================================

\section{research}

\begin{entrylist}
  \entry
    {2016--present}
    {Cooperation, Conflict, \& Cognition Lab}
    {Georgia State University}
    {Graduate Student Researcher (PI: Eyal Aharoni)}
  \entry
    {2014--present}
    {Language Research Center}
    {Georgia State University}
    {CEBUS Lab $\cdot$ Graduate Student Researcher (PI: Sarah Brosnan)}
  % \entry
  %   {2012--present}
  %   {MAD Insights $\cdot$ Making Adaptive Decisions}
  %   {}
  %   {Statistical Analyst \& Consultant}
  \entry
    {2012--2014}
    {Yerkes National Primate Research Center}
    {Emory University}
    {Living Links Center $\cdot$ Lab Manager $\cdot$ Chimpanzee Research Specialist}
  \entry
    {2011--2012}
    {Inkawu Vervet Project $\cdot$ South Africa}
    {U's of St Andrews, Neuch\^{a}tel, Z\"urich, \& Cape Town}
    {Field Research Assistant}
  \entry
    {2010--2011}
    {Wolfgang K\"{o}hler Primate Research Center}
    {Zoo Leipzig $\cdot$ Germany}
    {Research Assistant $\cdot$ Bachelor Student}
  \entry
    {2010--2011}
    {Educational Psychology Lab}
    {Martin Luther University $\cdot$ Germany}
    {Lab Coordinator}
  \entry
    {2010}
    {Child \& Adolescent Psychiatry \& Psychotherapy}
    {St Elisabeth Hospital Halle $\cdot$ Germany}
    {Intern}
  \entry
    {2009--2010}
    {Psychology of Perception Lab}
    {Martin Luther University $\cdot$ Germany}
    {Research Assistant}
  \entry
    {2009--2010}
    {Social Psychology Lab}
    {Martin Luther University $\cdot$ Germany}
    {Research Assistant}
\end{entrylist}%\\[-1cm] %<-- remove this when going on 4th page

% =================================================
% OUTREACH
% =================================================

% \section{outreach}

% \begin{entrylist}
%   \entry
%     {2015} % , 2017
%     {Brain Awareness Campaign}
%     {Society for Neuroscience}
%     {K-12 classroom visits $\cdot$ Atlanta, GA}
%   \entry
%     {2016}
%     {Adventure in Science Day}
%     {Fernbank Museum of Natural History}
%     {Science outreach for children $\cdot$ Atlanta, GA}
%   \entry
%     {2013}
%     {Nerd Nite}
%     {}
%     {Presentation $\cdot$ We are 98\,\% chimp and 50\,\% bananas: What apes can teach us about our own shortcomings $\cdot$ Atlanta, GA}
%   \entry
%     {2012--2014}
%     {Tour Guide}
%     {Emory University}
%     {Yerkes National Primate Research Center}
%   \entry
%     {2010--2011}
%     {Tour Guide}
%     {Zoo Leipzig $\cdot$ Germany}
%     {Wolfgang K\"{o}hler Primate Research Center}
% \end{entrylist}

% =================================================
% PROFESSIONAL SOCIETIES
% =================================================

% \section{professional societies}
%     Animal Behavior Society $\cdot$ %\\
%     American Society of Primatologists $\cdot$ %\\
%     APA Division 3: Society for Experimental Psychology and Cognitive Science $\cdot$ %\\
%     Association for Psychological Science $\cdot$ %\\
%     Comparative Cognition Society $\cdot$ %\\
%     International Society of Primatologists $\cdot$ %\\
%     Psi Chi $\cdot$ %\\
%     Psychonomic Society $\cdot$ %\\
%     Society for Computers in Psychology %\\[.5cm]

% =================================================
% AD HOC REVIEWER
% =================================================

\section{ad~hoc reviewer}
    American Journal of Primatology $\cdot$ Journal of Comparative Psychology \\[-.1cm]%[.5cm]


% \begin{aside}
%   \section{{\normalfont julia}watzek}
%     Dept of Psychology
%     Language Research Ctr
%     Georgia St University
%     ~
%     \href{mailto:j.watzek@gmail.com}{j.watzek@gmail.com}
%     +1 (404) 840-6838
% \end{aside}

\end{document}
